% Options for packages loaded elsewhere
\PassOptionsToPackage{unicode}{hyperref}
\PassOptionsToPackage{hyphens}{url}
%
\documentclass[
]{article}
\usepackage{lmodern}
\usepackage{amssymb,amsmath}
\usepackage{ifxetex,ifluatex}
\ifnum 0\ifxetex 1\fi\ifluatex 1\fi=0 % if pdftex
  \usepackage[T1]{fontenc}
  \usepackage[utf8]{inputenc}
  \usepackage{textcomp} % provide euro and other symbols
\else % if luatex or xetex
  \usepackage{unicode-math}
  \defaultfontfeatures{Scale=MatchLowercase}
  \defaultfontfeatures[\rmfamily]{Ligatures=TeX,Scale=1}
\fi
% Use upquote if available, for straight quotes in verbatim environments
\IfFileExists{upquote.sty}{\usepackage{upquote}}{}
\IfFileExists{microtype.sty}{% use microtype if available
  \usepackage[]{microtype}
  \UseMicrotypeSet[protrusion]{basicmath} % disable protrusion for tt fonts
}{}
\makeatletter
\@ifundefined{KOMAClassName}{% if non-KOMA class
  \IfFileExists{parskip.sty}{%
    \usepackage{parskip}
  }{% else
    \setlength{\parindent}{0pt}
    \setlength{\parskip}{6pt plus 2pt minus 1pt}}
}{% if KOMA class
  \KOMAoptions{parskip=half}}
\makeatother
\usepackage{xcolor}
\IfFileExists{xurl.sty}{\usepackage{xurl}}{} % add URL line breaks if available
\IfFileExists{bookmark.sty}{\usepackage{bookmark}}{\usepackage{hyperref}}
\hypersetup{
  pdftitle={Uncovering causality on TTC Line 2 Delays},
  pdfauthor={Elric Lazaro},
  hidelinks,
  pdfcreator={LaTeX via pandoc}}
\urlstyle{same} % disable monospaced font for URLs
\usepackage[margin=1in]{geometry}
\usepackage{graphicx,grffile}
\makeatletter
\def\maxwidth{\ifdim\Gin@nat@width>\linewidth\linewidth\else\Gin@nat@width\fi}
\def\maxheight{\ifdim\Gin@nat@height>\textheight\textheight\else\Gin@nat@height\fi}
\makeatother
% Scale images if necessary, so that they will not overflow the page
% margins by default, and it is still possible to overwrite the defaults
% using explicit options in \includegraphics[width, height, ...]{}
\setkeys{Gin}{width=\maxwidth,height=\maxheight,keepaspectratio}
% Set default figure placement to htbp
\makeatletter
\def\fps@figure{htbp}
\makeatother
\setlength{\emergencystretch}{3em} % prevent overfull lines
\providecommand{\tightlist}{%
  \setlength{\itemsep}{0pt}\setlength{\parskip}{0pt}}
\setcounter{secnumdepth}{-\maxdimen} % remove section numbering
\usepackage{booktabs}
\usepackage{longtable}
\usepackage{array}
\usepackage{multirow}
\usepackage{wrapfig}
\usepackage{float}
\usepackage{colortbl}
\usepackage{pdflscape}
\usepackage{tabu}
\usepackage{threeparttable}
\usepackage{threeparttablex}
\usepackage[normalem]{ulem}
\usepackage{makecell}

\title{Uncovering causality on TTC Line 2 Delays}
\author{Elric Lazaro}
\date{12/23/2020}

\begin{document}
\maketitle

\emph{Code and data supporting this analysis is available at:}\\
\url{https://github.com/ElricL/TTC-Line-2-Delay-Causality.git}

\hypertarget{abstract}{%
\subsection{Abstract}\label{abstract}}

~~~~~~This study seeks to find root causalities on TTC Line 2 Delays in
hopes of improving the subway system. Through the use of propensity
score matching, simulated subway entry data was analyzed based on travel
time. Evidence of cold seasonal times and peak days of the week were
found from observing the logistic regression model used to calculate the
propensity scores. However no significant findings were found when
evaluating the matching which suggests that a focus shift is required,
either by looking at the system more as a whole or diving in deeper.

\hypertarget{keywords}{%
\subsubsection{Keywords}\label{keywords}}

TTC, Toronto, Subway System, Propensity Score Matching, Delay, Causal
inference

\hypertarget{introduction}{%
\subsection{1. Introduction}\label{introduction}}

~~~~~~One of the many daily challenges students who did not live by UTSG
faced was commuting with major delays. This struggle can also be seen
from those that work in downtown Toronto as well. A Regular daily
commuting route often consists of many services such as MiWay buses, TTC
Subway, Go Buses, and many more. Yet all of which are very susceptible
to delays. TTC Subways often receive backlash, with the service often
requiring train exchanges, emergency stops, maintenance, and slowdowns.
Despite all the recent improvements such as Presto gates and more
accessibility elevators, to this day many delay issues are still
apparent with TTC subways. These problems have desensitized many TTC
Subway users who often simply leave home, school, or work early to beat
any possible major delays. While one paper may not be enough to fully
understand TTC's complex subway system, it is still important to start
uncovering some of the underlying problem's that we have come to accept
today.

~~~~~~This study seeks to delve deeper into what variables can likely
cause Subway delays and how the delays overall affect the subway user's
travel time, which remained largely unclear from previous study TTC
Subway Delay Cause Analysis (Lazaro, 2020). Propensity score matching
(PSM) will be used to identify any major causalities on Subway Delays on
TTC Line 2 stations. The advantage of PSM is that it will allow us to
compare observation outcomes and consider large number of variables
without it having a large effect on our sample size.

~~~~~~The observed data will be simulated based on observed patterns
from Toronto's open data on TTC Subway Delays along with the study, TTC
Subway Delay Cause Analysis (Lazaro, 2020). More on how the data is
simulated will be further expanded upon in the Methodology section.

\hypertarget{methodology}{%
\subsection{2. Methodology}\label{methodology}}

\hypertarget{data}{%
\subsubsection{2.1 Data}\label{data}}

~~~~~~The entries in the simulated data represents an individual
entering a station and their travel time onward. Each row or entry
consist of the month, day, station, bound (East or West), whether there
is a delay or not, and the travel time. Likelier events are prioritized
and have higher chance to appear in the simulated data. The target
population for this dataset is all TTC Subway users with the frame
population being the simulated users observed/generated in the dataset

~~~~~~Business days, school terms, and the reported busiest stations
from Urbanized (Chan, 2019) were taken into consideration into which
values are more likely to appear for month, day, and station. The sample
size totals to 500,000 observations given the simulated data is meant to
represent a year. Whether the individual is delayed or not depends on
the generated month, day, and station.

~~~~~~Utilizing the 2019 TTC Subway delays dataset observed from the
study, TTC Subway Delay Cause Analysis (Lazaro, 2020), I've ranked the
distinct values for the three variables based on the number of
occurrences. The rankings for each variables can be found in the
appendix with the lowest number having lowest number of occurrences to
highest number having highest number of occurrences. Note that the
values with number of occurrences that are tied or quite close to each
other will be assigned the same ranking. The rankings are applied for
each entries depending on what values are generated and then the total
ranking is used to calculate a softmax probability for the chance of
delay. Note bound was not used in this probability formula as both East
and West had very close number of occurrences. Since we used the rank
for month, day, and station to determine the probability of delay, we
are enforcing the more common occurrences observed from Toronto's
dataset to have higher chance on having delays in our simulation. For
instance we can see in Figure 1 that the simulated data relatively has
the higher numbered rank stations from the Toronto data to have highest
amount of delay occurrences.

\includegraphics{Uncovering-Causality-on-TTC-Delays_files/figure-latex/unnamed-chunk-3-1.pdf}
\textbf{Figure 1:} Top 10 Delay occurrences by Station

The softmax probability is calculated simply by
\(p = \frac{Sum\ of\ rankings}{exp(50)}\). Note that the probability of
not getting delayed would then be \(1 - p\).

~~~~~~Lastly to simulate the travel time, the normal distribution was
chosen. The normal distribution is used to take advantage on the
dataset's large size, specifically the sampling distribution of the mean
will approach closer to a normal distribution. The distribution applied
has a mean (\(\mu\)) 20 minutes for non delayed and 20 + 6.82 minutes
(travel time mean plus average delay time) when delayed. With the
average delay times per station from TTC Subway Delay Cause Analysis
(Lazaro, 2020), the mean delay of 6.82 was calculated. Check table 5 for
the average delay time by Month calculated from the previous study. As
for the standard mean, 20 was roughly estimated from the TTC Subway
Travel time Chart (Flack, 2019). To see examples of entries in the data,
please refer to Table 8 in the appendix.

~~~~~~With the simulated data generated, we now have our dataset which
we'll be studying through PSM methodology. The final two variables
generated are significant as delay will be our Treatment with travel
time being the outcome of interest for our PSM model which is further
explained in the 2.2 Model section.

\hypertarget{model}{%
\subsubsection{2.2 Model}\label{model}}

~~~~~~Propensity Score matching allows us to compare and evaluate the
outcome of our observations to determine any significant variables.
Firstly, we want to see the propensity of someone getting delayed and
then match based on that. A logistic regression model will be used to
determine the propensity score for each observations. Aside from
determining propensity scores, the model will also be used to observe
p-values of the independent variables based on delay to determine any
significant relationships. Specifically, check if the variables are less
than 0.05 to signify a relationship. The variable Delayed will be
modeled based on the independent variables Month, Day, Station, and
Bound:

\begin{align*}
log(\frac{(p)}{1-(p)}) &= {\beta}_0 + 
                            {\beta}_1X_{Month:\ January} + \\
                        &    {\beta}_2X_{Month:\ February} +
                            ... + \\
                        &    {\beta}_12X_{Month:\ December} + 
                            {\beta}_13X_{Day:\ Monday} + \\
                        &   {\beta}_14X_{Day:\ Tuesday} + 
                            ... + \\
                        &    {\beta}_19X_{Day:\ Sunday} + 
                            {\beta}_20X_{Station:\ KIPLING STATION} + \\
                        &   {\beta}_21X_{Station:\ ISLINGTON STATION} + 
                            ... + \\
                        &    {\beta}_{22}X_{Station:\ KENNEDY BD STATION} + 
                            {\beta}_{23}X_{Bound:\ E} + \\
                        &    {\beta}_{24}X_{Bound:\ W}
\end{align*}

Once the logistic model is calculated, it will be forecasted onto the
dataset, essentially calculating the probability of delay for each
individual's entry to the subway. A new dataset will then be created by
gathering those that have gotten delayed and match them with those that
have not gotten delayed but have similar or same propensity scores.

~~~~~~To determine how the delays and other observed variables affect
the travel time, a linear regression will be used on the new
reduced-matching dataset:

\begin{align*}
y &= {\beta}_0 + 
                            {\beta}_1X_{Month:\ January} + \\
                        &    {\beta}_2X_{Month:\ February} +
                            ... + \\
                        &    {\beta}_12X_{Month:\ December} + 
                            {\beta}_13X_{Day:\ Monday} + \\
                        &   {\beta}_14X_{Day:\ Tuesday} + 
                            ... + \\
                        &    {\beta}_19X_{Day:\ Sunday} + 
                            {\beta}_20X_{Station:\ KIPLING STATION} + \\
                        &   {\beta}_21X_{Station:\ ISLINGTON STATION} + 
                            ... + \\
                        &    {\beta}_{22}X_{Station:\ KENNEDY BD STATION} + 
                            {\beta}_{23}X_{Bound:\ E} + \\
                        &    {\beta}_{24}X_{Bound:\ W}  + {\beta}_{25}X_{Delayed}
\end{align*}

This model will allow us how the treatment variable, delay, will affect
the travel time as well as see any causalities with the other
independent variables. To discover any significant variables, we will
observe the p-value testing and seeing if it is less than 0.05 as a
result of the model.

\hypertarget{results}{%
\subsection{3. Results}\label{results}}

~~~~~~As a result of the the applying probabilities based on rankings
for the delay, in table 1 we can see there were 3,461 entries out of
500,000 observations that have experienced a delay. Calculating the
number of delay occurrences out of 500,000 observations, we can see in
theory there is approximately 0.7\% chance to experience delay every
time one enters a Line 2 subway station throughout the year.

\begin{table}[H]
\centering
\begin{tabular}{l|r}
\hline
Delayed & number of occurences\\
\hline
Not Delayed & 496539\\
\hline
Delayed & 3461\\
\hline
\end{tabular}
\end{table}

\textbf{Table 1:} Number of delay occurrences and those not delayed.

~~~~~~When applying the logistic regression model to determine
propensity scores, p-values for the independent variables based on delay
was observed. Looking at the values seen in Table 2, we can see that
some of the months such as December and each days of the week play a
large role in the likeliness of a delay occurrence. However with the
ranking probability logic applied, stations overall do not play much
significance on delays.

\begin{table}[H]
\centering
\begin{tabular}{l|r|r|r|r}
\hline
  & Estimate & Std. Error & z value & Pr(>|z|)\\
\hline
(Intercept) & -28.1378703 & 821.2133514 & -0.0342638 & 0.9726668\\
\hline
MonthAugust & -2.2739916 & 0.3923522 & -5.7957913 & 0.0000000\\
\hline
MonthDecember & -17.8810392 & 450.3680676 & -0.0397032 & 0.9683298\\
\hline
MonthFebruary & 2.3863526 & 0.1107212 & 21.5528119 & 0.0000000\\
\hline
MonthJanuary & 4.3113125 & 0.1106776 & 38.9538028 & 0.0000000\\
\hline
MonthJuly & -3.5636673 & 0.7148853 & -4.9849497 & 0.0000006\\
\hline
MonthJune & -3.1398109 & 0.5868599 & -5.3501882 & 0.0000001\\
\hline
MonthMarch & 1.1777352 & 0.1181936 & 9.9644600 & 0.0000000\\
\hline
MonthMay & -0.1342046 & 0.1674350 & -0.8015322 & 0.4228236\\
\hline
MonthNovember & -17.9430474 & 451.6330083 & -0.0397293 & 0.9683090\\
\hline
MonthOctober & -17.9245475 & 449.6940339 & -0.0398594 & 0.9682052\\
\hline
MonthSeptember & -3.9625573 & 0.7146958 & -5.5443967 & 0.0000000\\
\hline
DayMonday & -2.1548780 & 0.2092213 & -10.2995146 & 0.0000000\\
\hline
DaySaturday & -3.0374521 & 0.4557846 & -6.6642267 & 0.0000000\\
\hline
DaySunday & -3.8906363 & 0.7126303 & -5.4595440 & 0.0000000\\
\hline
DayThursday & 2.9916479 & 0.0891323 & 33.5641441 & 0.0000000\\
\hline
DayTuesday & 2.9577194 & 0.0892892 & 33.1251580 & 0.0000000\\
\hline
DayWednesday & -0.0567639 & 0.1088986 & -0.5212550 & 0.6021892\\
\hline
StationBAY STATION & 0.0076941 & 1168.9576768 & 0.0000066 & 0.9999947\\
\hline
StationBROADVIEW STATION & -0.0219590 & 1162.3688969 & -0.0000189 & 0.9999849\\
\hline
StationCASTLE FRANK STATION & 0.0564155 & 2405.4917353 & 0.0000235 & 0.9999813\\
\hline
StationCHESTER STATION & -0.0506982 & 2436.2323555 & -0.0000208 & 0.9999834\\
\hline
StationCHRISTIE STATION & -0.0347211 & 1168.6258582 & -0.0000297 & 0.9999763\\
\hline
StationCOXWELL STATION & 19.4044180 & 821.2133446 & 0.0236290 & 0.9811486\\
\hline
StationDONLANDS STATION & -0.0974270 & 2426.0715275 & -0.0000402 & 0.9999680\\
\hline
StationDUFFERIN STATION & -0.0313501 & 1159.5275143 & -0.0000270 & 0.9999784\\
\hline
StationDUNDAS WEST STATION & 0.0169916 & 1161.8818833 & 0.0000146 & 0.9999883\\
\hline
StationGREENWOOD STATION & 16.5028123 & 821.2139533 & 0.0200956 & 0.9839671\\
\hline
StationHIGH PARK STATION & 0.0654694 & 1168.4017118 & 0.0000560 & 0.9999553\\
\hline
StationISLINGTON STATION & 18.3585226 & 821.2133524 & 0.0223554 & 0.9821645\\
\hline
StationJANE STATION & 0.0310856 & 1164.5501904 & 0.0000267 & 0.9999787\\
\hline
StationKEELE STATION & 14.4907221 & 821.2139481 & 0.0176455 & 0.9859217\\
\hline
StationKENNEDY BD STATION & 20.5897700 & 821.2133395 & 0.0250724 & 0.9799972\\
\hline
StationKIPLING STATION & 22.5034064 & 821.2133396 & 0.0274026 & 0.9781386\\
\hline
StationLANSDOWNE STATION & -0.0025342 & 1165.7601416 & -0.0000022 & 0.9999983\\
\hline
StationMAIN STREET STATION & -0.0079423 & 1162.6582567 & -0.0000068 & 0.9999945\\
\hline
StationOLD MILL STATION & -0.1130151 & 2407.6330831 & -0.0000469 & 0.9999625\\
\hline
StationOSSINGTON STATION & 0.0645537 & 1168.5271581 & 0.0000552 & 0.9999559\\
\hline
StationPAPE STATION & 0.0185679 & 1166.0934231 & 0.0000159 & 0.9999873\\
\hline
StationROYAL YORK STATION & 0.0227634 & 1164.3965860 & 0.0000195 & 0.9999844\\
\hline
StationRUNNYMEDE STATION & -0.0323758 & 1159.9779772 & -0.0000279 & 0.9999777\\
\hline
StationSHERBOURNE STATION & 0.0083263 & 1163.0800374 & 0.0000072 & 0.9999943\\
\hline
StationSPADINA BD STATION & -0.0161599 & 1164.4617996 & -0.0000139 & 0.9999889\\
\hline
StationST GEORGE BD STATION & 13.4610277 & 821.2139475 & 0.0163916 & 0.9869220\\
\hline
StationVICTORIA PARK STATION & 17.1391024 & 821.2133827 & 0.0208705 & 0.9833490\\
\hline
StationWARDEN STATION & 17.4343252 & 821.2133712 & 0.0212300 & 0.9830622\\
\hline
StationWOODBINE STATION & 0.0093888 & 1162.1276513 & 0.0000081 & 0.9999936\\
\hline
StationYONGE BD STATION & 14.5558094 & 821.2135416 & 0.0177248 & 0.9858584\\
\hline
BoundW & -0.0398310 & 0.0461716 & -0.8626740 & 0.3883168\\
\hline
\end{tabular}
\end{table}

\textbf{Table 2:} Summary of Logistic Regression Model, used for
Propensity Score logic.

~~~~~~Given there are 3,461 delay occurrences, the reduced-matched
dataset consists of 6922 observations to be evaluated. A linear model
was used to determine relationships between the dependent variable
travel time and the independent variables, Month, Day, Station, Bound,
and Delayed. When looking at table 3, we can see most variables hold no
to little significance, due to having p-values greater than 0.05.
However, there appears to be a relationship to travel time and the month
of July with a negative coefficient of -2.37. Given a normal
distribution with different mean of 26.82 was applied for delay
occurrences during simulation (20 for non delays), there is a
relationship with travel time and delays with positive coefficient of
6.8.

\begin{table}[H]
\centering
\begin{tabular}{l|r|r|r|r}
\hline
  & Estimate & Std. Error & t value & Pr(>|t|)\\
\hline
(Intercept) & 19.9596623 & 0.2084609 & 95.7477621 & 0.0000000\\
\hline
MonthAugust & -0.1887109 & 0.5560649 & -0.3393684 & 0.7343426\\
\hline
MonthFebruary & 0.0167083 & 0.1479489 & 0.1129326 & 0.9100872\\
\hline
MonthJanuary & -0.0546206 & 0.1447816 & -0.3772621 & 0.7059904\\
\hline
MonthJuly & -2.3652837 & 1.0173328 & -2.3249851 & 0.0201016\\
\hline
MonthJune & -0.1620776 & 0.8339760 & -0.1943433 & 0.8459128\\
\hline
MonthMarch & 0.0708362 & 0.1616071 & 0.4383236 & 0.6611655\\
\hline
MonthMay & 0.0449783 & 0.2338650 & 0.1923258 & 0.8474927\\
\hline
MonthSeptember & -1.1748897 & 1.0168148 & -1.1554608 & 0.2479418\\
\hline
DayMonday & 0.2464894 & 0.2840986 & 0.8676193 & 0.3856330\\
\hline
DaySaturday & 0.5587904 & 0.6410926 & 0.8716220 & 0.3834450\\
\hline
DaySunday & -0.6821841 & 1.0094167 & -0.6758201 & 0.4991775\\
\hline
DayThursday & 0.0413585 & 0.0860442 & 0.4806657 & 0.6307694\\
\hline
DayTuesday & -0.0099921 & 0.0865511 & -0.1154476 & 0.9080937\\
\hline
DayWednesday & 0.1031579 & 0.1121133 & 0.9201219 & 0.3575412\\
\hline
StationGREENWOOD STATION & 0.4820181 & 1.4306980 & 0.3369112 & 0.7361941\\
\hline
StationISLINGTON STATION & -0.3036891 & 0.2486407 & -1.2213977 & 0.2219772\\
\hline
StationKEELE STATION & -0.0256381 & 1.4302212 & -0.0179260 & 0.9856984\\
\hline
StationKENNEDY BD STATION & 0.0480130 & 0.1409094 & 0.3407370 & 0.7333120\\
\hline
StationKIPLING STATION & 0.0259049 & 0.1374027 & 0.1885325 & 0.8504648\\
\hline
StationST GEORGE BD STATION & -2.6321906 & 1.4419764 & -1.8254048 & 0.0679832\\
\hline
StationVICTORIA PARK STATION & -0.1340740 & 0.4033664 & -0.3323877 & 0.7396066\\
\hline
StationWARDEN STATION & -0.1752085 & 0.3532713 & -0.4959603 & 0.6199382\\
\hline
StationYONGE BD STATION & -0.8542482 & 0.8331909 & -1.0252731 & 0.3052703\\
\hline
BoundW & -0.0183284 & 0.0488904 & -0.3748863 & 0.7077565\\
\hline
Delayed1 & 6.8002932 & 0.0518498 & 131.1536585 & 0.0000000\\
\hline
\end{tabular}
\end{table}

\textbf{Table 3:} Summary of Linear Regression Model, used for
evaluating propensity score matching.

\hypertarget{discussion}{%
\subsection{4 Discussion}\label{discussion}}

\hypertarget{summary}{%
\subsubsection{4.1 Summary}\label{summary}}

~~~~~~In this study, Line 2 subway station entries along with travel
time were simulated based on delay occurrence rankings for each variable
and travel time from TTC Travel time Chart (Flack, 2019). Propensity
score values are used to compare and match each similar observations.
The values are calculated by calculating a logistic regression model
from the simulated data and forecast it onto each observations. Entries
with delays are then matched to those without delays based on similar
propensity scores. The logistic model was also used to observe causality
on delays by observing the evaluated p-values. With the reduced dataset
with matched entries, we then examine the effect of getting delayed on
travel time through evaluating the linear regression model and examine
the p-values for any significant relationships.

\hypertarget{conclusion}{%
\subsubsection{4.2 Conclusion}\label{conclusion}}

~~~~~~If the simulated data is reproducible, the 0.7\% chance calculated
may generalize the percentage of delay occurrences. This means that
whenever one enters a Line 2 subway station intending to take the train,
one approximately has a chance of 0.7\% to experience a delay. Note this
is generalized for the entire year and may vary from month to month and
other aspects that have not been studied. While 0.7\% may seem small,
given that user such as a student or worker that commutes on a daily
basis, the chance of delay may no longer seem unlikely.

~~~~~~After having observed the p-values from the initial logistic
regression model, we can conclude that looking at each stations
independently holds no significance to the chance of delay. It appears
that the problem of delays cannot directly be connected to a specific
station. This means that an improvement cannot simply be made on a
station that may be lacking in performance rather one would need to look
at the system of line 2 subways as a whole. In contrast, we have
observed that weekdays and some of the months hold high significance.
For weekdays we can see that the weekends, Saturday and Sunday, have the
smallest chance of having delays given their small coefficients. Given
that there's less Subway users on the weekends, this could tell us that
the current system right now may not have the optimal support for large
amount of Subway users for Line 2. For the months that shows to have
significance with delays, one can observe that the months in winter
season have positive coefficients while those in other seasons have
negative. This supports the finding from previous study, TTC Subway
Delay Cause Analysis (Lazaro, 2020), that the Line 2 system may need
some improvement when it comes to colder weather.

~~~~~~With the propensity matching evaluated, there were no
relationships found for the outcome travel time aside from the variables
Delayed and the month of July. Delayed however holds no particular
interest given it is part of the process of simulating travel time as
observations that are delayed are enforced to have a longer travel time.
July has a negative coefficient which results in having faster travel
times. This may mean that the Subway in summer generally fares well in
comparison to other seasons. However given that other months had
p-values greater than 0.05, the relationship between July and travel
time remains inconclusive and may have appeared by chance due to the
dataset being simulated. While we may not have learned much causality
from conducting PSM, this could provide a lesson that a shift of focus
may be needed. For instance, it may be problematic that looking at the
stations independently given they are connected. It may be important to
look at the TTC system more as a whole or look at more specific aspects.

\hypertarget{weaknesses-and-next-steps}{%
\subsubsection{4.3 Weaknesses and Next
Steps}\label{weaknesses-and-next-steps}}

~~~~~~The data for this study is largely simulated, hence it may not
generalize well to an equivalent data gathered from a survey. The
probability for each value during the simulation process was evaluated
under assumptions and as well as the observations from another dataset.
Conducting studies on datasets related to the probabilities that were
assumed may result into a more accurate simulated table. While it is
very expensive it is also possible to conduct an experiment or a survey
for each station.

~~~~~~This study still uses Toronto's open data on TTC Subway delays for
reference on calculating the probability of delays. Unfortunately the
dataset contains uninterpretable values which had to be removed which
could result in skewing significant characteristics and creating bias
observations. Instead of removing, it is possible to contact staff
responsible of the dataset for consultance on how to clean up the
dataset.

~~~~~~Having only studied line 2, only a portion of the TTC subway
system was studied. Therefore this paper does not generalize well to the
overall system. This decision was made since each bounds have different
characteristics such as different trains and regions. Fortunately this
study can be reproduced for each line and we can then find any common
patterns between each study to generalize characteristics of the TTC
subway system as a whole.

~~~~~~In the end not many causalities were identified from PSM
methodology. This may be due to the methodology's problematic nature on
heavy reliance with complete randomization. This randomization may lead
to increase in imbalance. To see that this study's lack of causal
findings is a result of imbalance, we can check other matching methods
such as coarsened exact matching (King et al., 2019).

\hypertarget{appendix}{%
\subsection{5 Appendix}\label{appendix}}

\begin{table}[H]
\centering
\begin{tabular}{l|r|r}
\hline
Station & n & ranking\\
\hline
LANSDOWNE STATION & 74 & 1\\
\hline
RUNNYMEDE STATION & 74 & 2\\
\hline
DUFFERIN STATION & 86 & 3\\
\hline
PAPE STATION & 86 & 4\\
\hline
SHERBOURNE STATION & 89 & 5\\
\hline
CASTLE FRANK STATION & 94 & 6\\
\hline
CHESTER STATION & 101 & 7\\
\hline
BAY STATION & 106 & 8\\
\hline
SPADINA BD STATION & 115 & 9\\
\hline
BATHURST STATION & 117 & 10\\
\hline
MAIN STREET STATION & 117 & 11\\
\hline
HIGH PARK STATION & 120 & 12\\
\hline
DONLANDS STATION & 130 & 13\\
\hline
WOODBINE STATION & 133 & 14\\
\hline
CHRISTIE STATION & 141 & 15\\
\hline
BROADVIEW STATION & 147 & 16\\
\hline
OSSINGTON STATION & 147 & 17\\
\hline
DUNDAS WEST STATION & 162 & 18\\
\hline
OLD MILL STATION & 174 & 19\\
\hline
ROYAL YORK STATION & 188 & 20\\
\hline
JANE STATION & 208 & 21\\
\hline
ST GEORGE BD STATION & 218 & 22\\
\hline
YONGE BD STATION & 219 & 23\\
\hline
GREENWOOD STATION & 238 & 24\\
\hline
KEELE STATION & 241 & 25\\
\hline
VICTORIA PARK STATION & 305 & 26\\
\hline
WARDEN STATION & 342 & 27\\
\hline
ISLINGTON STATION & 379 & 28\\
\hline
COXWELL STATION & 402 & 29\\
\hline
KENNEDY BD STATION & 663 & 30\\
\hline
KIPLING STATION & 753 & 31\\
\hline
\end{tabular}
\end{table}

\textbf{Table 4:} Station rankings by number of delay occurrences. (From
Toronto's open TTC Delay Dataset)

\begin{table}[H]
\centering
\begin{tabular}{l|r|r|r}
\hline
Month & n & average\_delay\_time & ranking\\
\hline
November & 430 & 6.787645 & 1\\
\hline
December & 454 & 7.099222 & 2\\
\hline
October & 457 & 6.886827 & 3\\
\hline
September & 483 & 6.795494 & 4\\
\hline
June & 494 & 6.108153 & 5\\
\hline
July & 541 & 7.457539 & 6\\
\hline
August & 553 & 8.855856 & 7\\
\hline
April & 568 & 6.809240 & 8\\
\hline
May & 570 & 6.373494 & 9\\
\hline
March & 579 & 6.544423 & 10\\
\hline
February & 604 & 5.797320 & 11\\
\hline
January & 720 & 6.281098 & 12\\
\hline
\end{tabular}
\end{table}

\textbf{Table 5:} Month rankings by number of delay occurrences. Also
contains average delay time by month. (From Toronto's open TTC Delay
Dataset)

\begin{table}[H]
\centering
\begin{tabular}{l|r|r}
\hline
Day & n & ranking\\
\hline
Sunday & 640 & 1\\
\hline
Saturday & 763 & 2\\
\hline
Monday & 986 & 3\\
\hline
Friday & 1005 & 4\\
\hline
Wednesday & 1005 & 5\\
\hline
Thursday & 1026 & 6\\
\hline
Tuesday & 1028 & 7\\
\hline
\end{tabular}
\end{table}

\textbf{Table 6:} Day rankings by number of delay occurrences. (From
Toronto's open TTC Delay Dataset)

\begin{table}[H]
\centering
\begin{tabular}{l|r|r}
\hline
Bound & n & ranking\\
\hline
E & 3174 & 1\\
\hline
W & 3279 & 2\\
\hline
\end{tabular}
\end{table}

\textbf{Table 7:} Bound rankings by number of delay occurrences. (From
Toronto's open TTC Delay Dataset)

\begin{table}[H]
\centering
\begin{tabular}{r|l|l|l|l|l|r}
\hline
unique\_id & Month & Day & Station & Bound & Delayed & travel\_time\\
\hline
1 & August & Saturday & CHRISTIE STATION & W & 0 & 21.73145\\
\hline
3 & December & Sunday & SHERBOURNE STATION & E & 0 & 19.55909\\
\hline
10 & November & Sunday & DUNDAS WEST STATION & E & 0 & 20.79242\\
\hline
12 & June & Friday & PAPE STATION & W & 0 & 22.44426\\
\hline
15 & August & Wednesday & OSSINGTON STATION & W & 0 & 16.18878\\
\hline
16 & December & Thursday & OSSINGTON STATION & E & 0 & 21.08448\\
\hline
\end{tabular}
\end{table}

\textbf{Table 8:} Sample of the simulated station entry dataset.

\hypertarget{references}{%
\subsection{References}\label{references}}

Chan, K. (2019, August 08). Urbanized. Retrieved December 23, 2021, from
\url{https://dailyhive.com/toronto/ttc-toronto-subway-station-ridership-2018}

Flack, D. (2019, May 11). These are the ideal travel times between TTC
subway stations. Retrieved December 23, 2021, from
\url{https://www.blogto.com/city/2017/02/ideal-travel-times-between-ttc-subway-stations/}

TTC. (2014, December). TTC Operating Statistics. Retrieved December 23,
2021, from \url{https://www.ttc.ca/Coupler/Short_Turns/Operating}
Statistics/index.jsp

King, G., \& Nielsen, R. (2019). Why Propensity Scores Should Not Be
Used for Matching. Political Analysis, 27(4), 435-454.
\url{doi:10.1017/pan.2019.11}

Lazaro, E. (2020). TTC Subway Delay Cause Analysis. TTC Subway Delay
Cause Analysis.

\end{document}
